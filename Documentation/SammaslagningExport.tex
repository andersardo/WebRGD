\documentclass[swedish,11pt]{article}
\usepackage[english]{babel}
\usepackage[utf8x]{inputenc}
\usepackage{times}
\usepackage[T1]{fontenc}
\usepackage{tikz}
\usetikzlibrary{arrows,snakes,backgrounds,shapes}
\usepackage{hyperref}
\textwidth=16cm
\textheight=23cm
\oddsidemargin=-0.1cm
\evensidemargin=-0.1cm
\topmargin=-2cm

\title{openRGD - sammanslagning och Gedcom-export \newline
  Version 1.1}
\author{Anders Ardö\\Anders.Ardo@gmail.com}

\begin{document}
\maketitle

\section{Inledning/Översikt}
openRGD använder (och sparar i databasen) endast namn, född, död, gift samt relationer
- resten av information sparas i orginalform som ett Gedcom-fragment.

Målet för produktions RGD är att behandla nedanstående uppgifter/Gedcom taggar enligt tabellen.
Kolumnen 'matchning' indikerar om en uppgift ska användas vid matchningen.
Så mycket som möjligt av detta ska implementeras i openRGD.
Kolumnen 'openRGD' anger
var i processen (jmfr UI) detta sker eller är tänkt att ske.

%När det gäller export av NAME, BIRT, DEAT, MARR till Gedcom så exporteras först
%den sammanslagna uppgiften,
%sedan görs ett försök att exportera även sådana uppgifter där data inte stämmer överens med
%den sammanslagna uppgiften.
%<Koden för detta ser konstig ut!!>

\begin{tabular}{|l|l|l|l|l|l|}
  \hline
 Uppgift & Gedcom & typ & matchning & sammanslagning & openRGD \\ \hline \hline
namn & NAME  & event & * & slås samman & 7. Sammanslagning \\ \hline
kön & SEX  & event & * & slås samman & 7. Sammanslagning \\ \hline
född &  BIRT & event & * & slås samman & 7. Sammanslagning \\ \hline
döpt &  CHR  & event & BIRT saknas & slås samman & 8. Skapa Gedcom \\ \hline
gift &    MARR & event & * & slås samman & 7. Sammanslagning \\ \hline
död  &  DEAT & event & * & slås samman & 7. Sammanslagning \\ \hline
begravd & BURI & event & DEAT saknas & slås samman & 8. Skapa Gedcom \\ \hline
flyttad & - & & & & \\ \hline
inflyttad & IMMI & event & & alla unika sparas & 8. Skapa Gedcom \\ \hline
utflyttad & EMIG & event & & alla unika sparas & 8. Skapa Gedcom \\ \hline
levde & RESI & event & & alla unika  sparas & 8. Skapa Gedcom \\ \hline
text &  NOTE & text & &  alla unika  sparas & 8. Skapa Gedcom \\ \hline
special & NOTE & & &  alla unika  sparas & 8. Skapa Gedcom \\ \hline
yrke & OCCU & & &  alla unika  sparas & 8. Skapa Gedcom \\ \hline
skild & DIV & event & & alla unika sparas & 8. Skapa Gedcom \\ \hline
händelse & EVEN & event & & alla unika sparas & 8. Skapa Gedcom \\ \hline
 & alla andra & & & alla sparas & 8. Skapa Gedcom \\ \hline
\end{tabular}

\subsection{Status}
Huvuddelen av det som behandlas i detta dokument (inklusive frågorna nedan och algoritm \ref{algoritm})
är implementerat, men inte uppladdat till DIS server. 
Ett undatag är kommentaren (av Anders H)
om viktning/prioritering i avsnitt \ref{redmine1943}.
Koden är inte heller testad speciellt mycket.

\section{Frågor}
\subsection{TYPE i event}
Vissa händelser (event) innehåller förutom DATE, PLAC, och SOUR även en TYPE
- ska den behandlas som de andra?

Redmine RGD Design 1211 anger att det kan finnas en massa andra taggar också.

Följande Gedcom-taggar har förekommit i BIRT, DEAT, MARR events i filer i RGD:\\
CONT,
\_PLC,
TEXT,
\_PRIM,
CONC,
FILE,
DATE,
\_SIZE,
DATA,
\_CONF\_NOTE,
ADR1,
\_PRNT,
\_SCBK,
TITL,
\_STYLE,
LONG,
\_NAME,
CAUS,
SOUR,
PHON,
\_TYPE,
QUAY,
FORM,
CITY,
\_APID,
OBJE,
PLAC,
POST,
MAP,
ADDR,
TYPE,
NOTE,
LATI,
PAGE

{\it Kalle:}
\begin{quotation}
  \label{type}
TYPE bör absolut inkluderas. Den innehåller förklarande information, t.ex. före eller utom äktenskap vör en relation. Men TYPE kan också innehålla tillägg som släktforskaren gjort för att förtydliga.
Taggen EVEN skall alltid har TYPE för att tala om vilken typ av händelse det är.

En händelse kan bara ha max 1 TYPE
\end{quotation}

\subsubsection{Vid sammanslagning av en event ska taggarna DATE, PLAC, SOUR, och TYPE tas hand om.}

\subsection{Redmine RGD Design 3514}

Vid sammanslagning av händelser kan man komma att välja DATE från en händelse (A)
och PLAC från en händelse (B). Vad ska då SOUR för den sammanslagna händelsen sättas till?
\begin{enumerate}
\item SOUR från A
\item SOUR från B
\item 2 st SOUR (från A och B)
\item SOUR = RGD med en NOTE som anger SOUR A och SOUR B
\end{enumerate}

Hur generaliserar man detta till sammanslagning av N händelser?

{\it Kalle:}
\begin{quotation}
PLAC och SOUR bör hänga samman. SOUR i sig själv ger inte alltid en komplett källa men PLAC och SOUR tillsammans skall alltid göra det. Detta gäller främst när kyrkböcker används som källa.

Datum skulle kunna tas från den andra posten om datumet är mer komplett. Som i ett exempel du tidigare visade att posten med bästa källan bara hade årtal men att den andra posten hade komplett datum.
\end{quotation}

\subsubsection{Se avsnitt \ref{algoritm}}

\subsection{Redmine RGD Design 1943 + mail - Välja data vid sammanslagning}
\label{redmine1943}
Källan (SOUR) kan eventuellt, efter maskinell kvalitetsbedömning börja med asterisk och en siffra
Maskinellt *1 till *9, förbered för manuellt satt *0 som bästa värde.

Om de poster som skall mergas båda har ett visst data skall:
\begin{itemize}
\item källan med lägsta siffra efter asterisken användas
\item om bara en källa har asterisk, använd den
\item om båda har samma värde efter asterisken används den med mest kompletta texten
\item om båda saknar asterisk används den med mest kompletta texten
\end{itemize}
  ({\it hur ska 'mest kompletta texten' tolkas maskinellt? Den längsta?})

{\it Kalle:}
\begin{quotation}
Mest komplett text är troligen inte alltid den bästa men det är svårt att sätta en regel.
Om uppgifterna i övrigt, datum och ort, är lika kanske förslaget från punkt 4 i 2.2 användas.
Är datum och ort i konflikt kanske man kan använda principen att det är den filen man gör sammanslagningen till som "vinner". Datum och plats på matchade individer bör dock inte släppas igenom manuella matchningen.
\end{quotation}

{\it Anders H:}
\begin{quotation}
En ytterligare viktning, som jag tror nämndes någonstans, är att den databas man matchar TILL, har en något högre prioritet vi konflikter.
\end{quotation}

\subsubsection{Algoritm för att välja data vid sammanslagning av händelser (event)}
Sammanslagning av databaser B och A till databas A (se också \ref{redmine1943}).
\label{algoritm}

Välj {\bf PLAC och SOUR} från den händelse som har högst kvalitet.

Vid flera likvärdiga och händelsen från A tillhör dessa välj {\bf PLAC och SOUR} från A.

Annars välj {\bf PLAC och SOUR} slumpmässigt bland de händelser som har högst kvalitet.

Välj det mest kompletta {\bf DATE}.

Vid flera likvärdiga välj {\bf DATE} bland de med högst kvalitet (med prioritet för händelsen från A).

Texten från alla unika {\bf TYPE} adderas till en samlad TYPE rad.

Övriga taggar ignoreras.

\subsection{Vilken källa ska anges}
Vi har inte bestämt vilken källa som kan/skall användas i utfilen.
 
Original källa så som den skrivits i den “bästa” källan (detta är väl normalläget)
 
Den manipulerade källan som eventuellt innehåller Asterisk och nivåvärde.
Den innehåller också vissa förkortningar expanderade på ett standardiserat sätt.

\subsubsection{Förslag}
Inkludera bägge: Orginal källa som SOUR och manipulerad källa som NOTE.

\subsection{Redmine RGD Design 3511 -Ursprungsmarkering vid sammanslagning}


För att poster vid sammanslagning skall kunna identifieras av användaren finns en NOTE tagg.

Denna information bör bli mer informativ för att passa vid sammanslagning av flera filer.
Förslag:

Original följt av namnet på filen och som nu identiteten på individen
xempel:\\
1 NOTE Original Kalle 26-10457\\
1 NOTE Original Gun 14-190

Ger spårbarhet till vem som bidragit till posten och den identiteten i den personens släktforskning.
Är det ingen sammanslagen post finns också motsvarande information

\end{document}
